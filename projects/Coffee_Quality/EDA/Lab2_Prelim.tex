% Options for packages loaded elsewhere
\PassOptionsToPackage{unicode}{hyperref}
\PassOptionsToPackage{hyphens}{url}
%
\documentclass[
]{article}
\usepackage{amsmath,amssymb}
\usepackage{lmodern}
\usepackage{ifxetex,ifluatex}
\ifnum 0\ifxetex 1\fi\ifluatex 1\fi=0 % if pdftex
  \usepackage[T1]{fontenc}
  \usepackage[utf8]{inputenc}
  \usepackage{textcomp} % provide euro and other symbols
\else % if luatex or xetex
  \usepackage{unicode-math}
  \defaultfontfeatures{Scale=MatchLowercase}
  \defaultfontfeatures[\rmfamily]{Ligatures=TeX,Scale=1}
\fi
% Use upquote if available, for straight quotes in verbatim environments
\IfFileExists{upquote.sty}{\usepackage{upquote}}{}
\IfFileExists{microtype.sty}{% use microtype if available
  \usepackage[]{microtype}
  \UseMicrotypeSet[protrusion]{basicmath} % disable protrusion for tt fonts
}{}
\makeatletter
\@ifundefined{KOMAClassName}{% if non-KOMA class
  \IfFileExists{parskip.sty}{%
    \usepackage{parskip}
  }{% else
    \setlength{\parindent}{0pt}
    \setlength{\parskip}{6pt plus 2pt minus 1pt}}
}{% if KOMA class
  \KOMAoptions{parskip=half}}
\makeatother
\usepackage{xcolor}
\IfFileExists{xurl.sty}{\usepackage{xurl}}{} % add URL line breaks if available
\IfFileExists{bookmark.sty}{\usepackage{bookmark}}{\usepackage{hyperref}}
\hypersetup{
  pdftitle={W203 Lab 2 Preliminary Report},
  pdfauthor={Angela Guan, Amy Jung, Jeremy Yeung},
  hidelinks,
  pdfcreator={LaTeX via pandoc}}
\urlstyle{same} % disable monospaced font for URLs
\usepackage[margin=1in]{geometry}
\usepackage{color}
\usepackage{fancyvrb}
\newcommand{\VerbBar}{|}
\newcommand{\VERB}{\Verb[commandchars=\\\{\}]}
\DefineVerbatimEnvironment{Highlighting}{Verbatim}{commandchars=\\\{\}}
% Add ',fontsize=\small' for more characters per line
\usepackage{framed}
\definecolor{shadecolor}{RGB}{248,248,248}
\newenvironment{Shaded}{\begin{snugshade}}{\end{snugshade}}
\newcommand{\AlertTok}[1]{\textcolor[rgb]{0.94,0.16,0.16}{#1}}
\newcommand{\AnnotationTok}[1]{\textcolor[rgb]{0.56,0.35,0.01}{\textbf{\textit{#1}}}}
\newcommand{\AttributeTok}[1]{\textcolor[rgb]{0.77,0.63,0.00}{#1}}
\newcommand{\BaseNTok}[1]{\textcolor[rgb]{0.00,0.00,0.81}{#1}}
\newcommand{\BuiltInTok}[1]{#1}
\newcommand{\CharTok}[1]{\textcolor[rgb]{0.31,0.60,0.02}{#1}}
\newcommand{\CommentTok}[1]{\textcolor[rgb]{0.56,0.35,0.01}{\textit{#1}}}
\newcommand{\CommentVarTok}[1]{\textcolor[rgb]{0.56,0.35,0.01}{\textbf{\textit{#1}}}}
\newcommand{\ConstantTok}[1]{\textcolor[rgb]{0.00,0.00,0.00}{#1}}
\newcommand{\ControlFlowTok}[1]{\textcolor[rgb]{0.13,0.29,0.53}{\textbf{#1}}}
\newcommand{\DataTypeTok}[1]{\textcolor[rgb]{0.13,0.29,0.53}{#1}}
\newcommand{\DecValTok}[1]{\textcolor[rgb]{0.00,0.00,0.81}{#1}}
\newcommand{\DocumentationTok}[1]{\textcolor[rgb]{0.56,0.35,0.01}{\textbf{\textit{#1}}}}
\newcommand{\ErrorTok}[1]{\textcolor[rgb]{0.64,0.00,0.00}{\textbf{#1}}}
\newcommand{\ExtensionTok}[1]{#1}
\newcommand{\FloatTok}[1]{\textcolor[rgb]{0.00,0.00,0.81}{#1}}
\newcommand{\FunctionTok}[1]{\textcolor[rgb]{0.00,0.00,0.00}{#1}}
\newcommand{\ImportTok}[1]{#1}
\newcommand{\InformationTok}[1]{\textcolor[rgb]{0.56,0.35,0.01}{\textbf{\textit{#1}}}}
\newcommand{\KeywordTok}[1]{\textcolor[rgb]{0.13,0.29,0.53}{\textbf{#1}}}
\newcommand{\NormalTok}[1]{#1}
\newcommand{\OperatorTok}[1]{\textcolor[rgb]{0.81,0.36,0.00}{\textbf{#1}}}
\newcommand{\OtherTok}[1]{\textcolor[rgb]{0.56,0.35,0.01}{#1}}
\newcommand{\PreprocessorTok}[1]{\textcolor[rgb]{0.56,0.35,0.01}{\textit{#1}}}
\newcommand{\RegionMarkerTok}[1]{#1}
\newcommand{\SpecialCharTok}[1]{\textcolor[rgb]{0.00,0.00,0.00}{#1}}
\newcommand{\SpecialStringTok}[1]{\textcolor[rgb]{0.31,0.60,0.02}{#1}}
\newcommand{\StringTok}[1]{\textcolor[rgb]{0.31,0.60,0.02}{#1}}
\newcommand{\VariableTok}[1]{\textcolor[rgb]{0.00,0.00,0.00}{#1}}
\newcommand{\VerbatimStringTok}[1]{\textcolor[rgb]{0.31,0.60,0.02}{#1}}
\newcommand{\WarningTok}[1]{\textcolor[rgb]{0.56,0.35,0.01}{\textbf{\textit{#1}}}}
\usepackage{longtable,booktabs,array}
\usepackage{calc} % for calculating minipage widths
% Correct order of tables after \paragraph or \subparagraph
\usepackage{etoolbox}
\makeatletter
\patchcmd\longtable{\par}{\if@noskipsec\mbox{}\fi\par}{}{}
\makeatother
% Allow footnotes in longtable head/foot
\IfFileExists{footnotehyper.sty}{\usepackage{footnotehyper}}{\usepackage{footnote}}
\makesavenoteenv{longtable}
\usepackage{graphicx}
\makeatletter
\def\maxwidth{\ifdim\Gin@nat@width>\linewidth\linewidth\else\Gin@nat@width\fi}
\def\maxheight{\ifdim\Gin@nat@height>\textheight\textheight\else\Gin@nat@height\fi}
\makeatother
% Scale images if necessary, so that they will not overflow the page
% margins by default, and it is still possible to overwrite the defaults
% using explicit options in \includegraphics[width, height, ...]{}
\setkeys{Gin}{width=\maxwidth,height=\maxheight,keepaspectratio}
% Set default figure placement to htbp
\makeatletter
\def\fps@figure{htbp}
\makeatother
\setlength{\emergencystretch}{3em} % prevent overfull lines
\providecommand{\tightlist}{%
  \setlength{\itemsep}{0pt}\setlength{\parskip}{0pt}}
\setcounter{secnumdepth}{-\maxdimen} % remove section numbering
\ifluatex
  \usepackage{selnolig}  % disable illegal ligatures
\fi

\title{W203 Lab 2 Preliminary Report}
\author{Angela Guan, Amy Jung, Jeremy Yeung}
\date{}

\begin{document}
\maketitle

\begin{Shaded}
\begin{Highlighting}[]
\FunctionTok{library}\NormalTok{(tidyverse)}
\FunctionTok{library}\NormalTok{(ggplot2) }

\FunctionTok{library}\NormalTok{(sandwich)}
\FunctionTok{library}\NormalTok{(stargazer)}

\FunctionTok{source}\NormalTok{(}\StringTok{\textquotesingle{}get\_robust\_se.R\textquotesingle{}}\NormalTok{)}
\end{Highlighting}
\end{Shaded}

\begin{Shaded}
\begin{Highlighting}[]
\NormalTok{coffee\_ratings }\OtherTok{\textless{}{-}}\NormalTok{ readr}\SpecialCharTok{::}\FunctionTok{read\_csv}\NormalTok{(}\StringTok{\textquotesingle{}https://raw.githubusercontent.com/rfordatascience/tidytuesday/master/data/2020/2020{-}07{-}07/coffee\_ratings.csv\textquotesingle{}}\NormalTok{)}
\CommentTok{\#head(coffee\_ratings)}
\end{Highlighting}
\end{Shaded}

\begin{verbatim}
## Warning: The following named parsers don't match the column names: X1
\end{verbatim}

\begin{longtable}[]{@{}ll@{}}
\caption{Data summary}\tabularnewline
\toprule
& \\
\midrule
\endfirsthead
\toprule
& \\
\midrule
\endhead
Name & Piped data \\
Number of rows & 1339 \\
Number of columns & 43 \\
\_\_\_\_\_\_\_\_\_\_\_\_\_\_\_\_\_\_\_\_\_\_\_ & \\
Column type frequency: & \\
character & 24 \\
numeric & 19 \\
\_\_\_\_\_\_\_\_\_\_\_\_\_\_\_\_\_\_\_\_\_\_\_\_ & \\
Group variables & None \\
\bottomrule
\end{longtable}

\textbf{Variable type: character}

\begin{longtable}[]{@{}lrrrrrrr@{}}
\toprule
skim\_variable & n\_missing & complete\_rate & min & max & empty &
n\_unique & whitespace \\
\midrule
\endhead
species & 0 & 1.00 & 7 & 7 & 0 & 2 & 0 \\
owner & 7 & 0.99 & 3 & 50 & 0 & 315 & 0 \\
country\_of\_origin & 1 & 1.00 & 4 & 28 & 0 & 36 & 0 \\
farm\_name & 359 & 0.73 & 1 & 73 & 0 & 571 & 0 \\
lot\_number & 1063 & 0.21 & 1 & 71 & 0 & 227 & 0 \\
mill & 315 & 0.76 & 1 & 77 & 0 & 460 & 0 \\
ico\_number & 151 & 0.89 & 1 & 40 & 0 & 847 & 0 \\
company & 209 & 0.84 & 3 & 73 & 0 & 281 & 0 \\
altitude & 226 & 0.83 & 1 & 41 & 0 & 396 & 0 \\
region & 59 & 0.96 & 2 & 76 & 0 & 356 & 0 \\
producer & 231 & 0.83 & 1 & 100 & 0 & 691 & 0 \\
bag\_weight & 0 & 1.00 & 1 & 8 & 0 & 56 & 0 \\
in\_country\_partner & 0 & 1.00 & 7 & 85 & 0 & 27 & 0 \\
harvest\_year & 47 & 0.96 & 3 & 24 & 0 & 46 & 0 \\
grading\_date & 0 & 1.00 & 13 & 20 & 0 & 567 & 0 \\
owner\_1 & 7 & 0.99 & 3 & 50 & 0 & 319 & 0 \\
variety & 226 & 0.83 & 4 & 21 & 0 & 29 & 0 \\
processing\_method & 170 & 0.87 & 5 & 25 & 0 & 5 & 0 \\
color & 218 & 0.84 & 4 & 12 & 0 & 4 & 0 \\
expiration & 0 & 1.00 & 13 & 20 & 0 & 566 & 0 \\
certification\_body & 0 & 1.00 & 7 & 85 & 0 & 26 & 0 \\
certification\_address & 0 & 1.00 & 40 & 40 & 0 & 32 & 0 \\
certification\_contact & 0 & 1.00 & 40 & 40 & 0 & 29 & 0 \\
unit\_of\_measurement & 0 & 1.00 & 1 & 2 & 0 & 2 & 0 \\
\bottomrule
\end{longtable}

\textbf{Variable type: numeric}

\begin{longtable}[]{@{}
  >{\raggedright\arraybackslash}p{(\columnwidth - 20\tabcolsep) * \real{0.20}}
  >{\raggedleft\arraybackslash}p{(\columnwidth - 20\tabcolsep) * \real{0.10}}
  >{\raggedleft\arraybackslash}p{(\columnwidth - 20\tabcolsep) * \real{0.13}}
  >{\raggedleft\arraybackslash}p{(\columnwidth - 20\tabcolsep) * \real{0.08}}
  >{\raggedleft\arraybackslash}p{(\columnwidth - 20\tabcolsep) * \real{0.08}}
  >{\raggedleft\arraybackslash}p{(\columnwidth - 20\tabcolsep) * \real{0.03}}
  >{\raggedleft\arraybackslash}p{(\columnwidth - 20\tabcolsep) * \real{0.08}}
  >{\raggedleft\arraybackslash}p{(\columnwidth - 20\tabcolsep) * \real{0.08}}
  >{\raggedleft\arraybackslash}p{(\columnwidth - 20\tabcolsep) * \real{0.08}}
  >{\raggedleft\arraybackslash}p{(\columnwidth - 20\tabcolsep) * \real{0.10}}
  >{\raggedright\arraybackslash}p{(\columnwidth - 20\tabcolsep) * \real{0.06}}@{}}
\toprule
skim\_variable & n\_missing & complete\_rate & mean & sd & p0 & p25 &
p50 & p75 & p100 & hist \\
\midrule
\endhead
total\_cup\_points & 0 & 1.00 & 82.09 & 3.50 & 0 & 81.08 & 82.50 & 83.67
& 90.58 & ▁▁▁▁▇ \\
number\_of\_bags & 0 & 1.00 & 154.18 & 129.99 & 0 & 14.00 & 175.00 &
275.00 & 1062.00 & ▇▇▁▁▁ \\
aroma & 0 & 1.00 & 7.57 & 0.38 & 0 & 7.42 & 7.58 & 7.75 & 8.75 &
▁▁▁▁▇ \\
flavor & 0 & 1.00 & 7.52 & 0.40 & 0 & 7.33 & 7.58 & 7.75 & 8.83 &
▁▁▁▁▇ \\
aftertaste & 0 & 1.00 & 7.40 & 0.40 & 0 & 7.25 & 7.42 & 7.58 & 8.67 &
▁▁▁▁▇ \\
acidity & 0 & 1.00 & 7.54 & 0.38 & 0 & 7.33 & 7.58 & 7.75 & 8.75 &
▁▁▁▁▇ \\
body & 0 & 1.00 & 7.52 & 0.37 & 0 & 7.33 & 7.50 & 7.67 & 8.58 & ▁▁▁▁▇ \\
balance & 0 & 1.00 & 7.52 & 0.41 & 0 & 7.33 & 7.50 & 7.75 & 8.75 &
▁▁▁▁▇ \\
uniformity & 0 & 1.00 & 9.83 & 0.55 & 0 & 10.00 & 10.00 & 10.00 & 10.00
& ▁▁▁▁▇ \\
clean\_cup & 0 & 1.00 & 9.84 & 0.76 & 0 & 10.00 & 10.00 & 10.00 & 10.00
& ▁▁▁▁▇ \\
sweetness & 0 & 1.00 & 9.86 & 0.62 & 0 & 10.00 & 10.00 & 10.00 & 10.00 &
▁▁▁▁▇ \\
cupper\_points & 0 & 1.00 & 7.50 & 0.47 & 0 & 7.25 & 7.50 & 7.75 & 10.00
& ▁▁▁▇▁ \\
moisture & 0 & 1.00 & 0.09 & 0.05 & 0 & 0.09 & 0.11 & 0.12 & 0.28 &
▃▇▅▁▁ \\
category\_one\_defects & 0 & 1.00 & 0.48 & 2.55 & 0 & 0.00 & 0.00 & 0.00
& 63.00 & ▇▁▁▁▁ \\
quakers & 1 & 1.00 & 0.17 & 0.83 & 0 & 0.00 & 0.00 & 0.00 & 11.00 &
▇▁▁▁▁ \\
category\_two\_defects & 0 & 1.00 & 3.56 & 5.31 & 0 & 0.00 & 2.00 & 4.00
& 55.00 & ▇▁▁▁▁ \\
altitude\_low\_meters & 230 & 0.83 & 1750.71 & 8669.44 & 1 & 1100.00 &
1310.64 & 1600.00 & 190164.00 & ▇▁▁▁▁ \\
altitude\_high\_meters & 230 & 0.83 & 1799.35 & 8668.81 & 1 & 1100.00 &
1350.00 & 1650.00 & 190164.00 & ▇▁▁▁▁ \\
altitude\_mean\_meters & 230 & 0.83 & 1775.03 & 8668.63 & 1 & 1100.00 &
1310.64 & 1600.00 & 190164.00 & ▇▁▁▁▁ \\
\bottomrule
\end{longtable}

\begin{Shaded}
\begin{Highlighting}[]
\CommentTok{\#head(all\_ratings)}
\end{Highlighting}
\end{Shaded}

\begin{Shaded}
\begin{Highlighting}[]
\CommentTok{\#nrow(all\_ratings)}
\end{Highlighting}
\end{Shaded}

\begin{Shaded}
\begin{Highlighting}[]
\CommentTok{\# keep relevant columns}
\NormalTok{subset\_coffee }\OtherTok{=}\NormalTok{ all\_ratings[}\FunctionTok{c}\NormalTok{(}\StringTok{"total\_cup\_points"}\NormalTok{, }\StringTok{"region"}\NormalTok{, }\StringTok{"processing\_method"}\NormalTok{, }\StringTok{"aroma"}\NormalTok{, }\StringTok{"aftertaste"}\NormalTok{, }\StringTok{"acidity"}\NormalTok{, }\StringTok{"body"}\NormalTok{, }\StringTok{"balance"}\NormalTok{, }\StringTok{"sweetness"}\NormalTok{, }\StringTok{"color"}\NormalTok{)]}
\CommentTok{\#head(subset\_coffee)}
\end{Highlighting}
\end{Shaded}

\begin{Shaded}
\begin{Highlighting}[]
\CommentTok{\# about 400 NAs so we will drop them}
\FunctionTok{sum}\NormalTok{(}\FunctionTok{is.na}\NormalTok{(subset\_coffee))}
\end{Highlighting}
\end{Shaded}

\begin{verbatim}
## [1] 447
\end{verbatim}

\begin{Shaded}
\begin{Highlighting}[]
\NormalTok{cleaned\_coffee }\OtherTok{=} \FunctionTok{drop\_na}\NormalTok{(subset\_coffee)}
\CommentTok{\#head(cleaned\_coffee)}
\end{Highlighting}
\end{Shaded}

\begin{Shaded}
\begin{Highlighting}[]
\CommentTok{\#nrow(cleaned\_coffee)}
\end{Highlighting}
\end{Shaded}

\begin{Shaded}
\begin{Highlighting}[]
\CommentTok{\#ggplot(cleaned\_coffee, aes(x=total\_cup\_points)) + geom\_histogram() + labs(title="distribution of total cup points")}
\CommentTok{\#print("top 5 regions:")}
\CommentTok{\#sort(table(cleaned\_coffee$region), decreasing=TRUE)[1:5]}
\CommentTok{\#print("most common processing methods:")}
\CommentTok{\#sort(table(cleaned\_coffee$processing\_method), decreasing=TRUE)}
\CommentTok{\#ggplot(cleaned\_coffee, aes(x=aroma)) + geom\_histogram() + labs(title="distribution of aroma")}
\CommentTok{\#ggplot(cleaned\_coffee, aes(x=aftertaste)) + geom\_histogram() + labs(title="distribution of aftertaste")}
\CommentTok{\#ggplot(cleaned\_coffee, aes(x=acidity)) + geom\_histogram() + labs(title="distribution of acidity")}
\CommentTok{\#ggplot(cleaned\_coffee, aes(x=body)) + geom\_histogram() + labs(title="distribution of body")}
\CommentTok{\#ggplot(cleaned\_coffee, aes(x=balance)) + geom\_histogram() + labs(title="distribution of balance")}
\CommentTok{\#ggplot(cleaned\_coffee, aes(x=sweetness)) + geom\_histogram() + labs(title="distribution of sweetness")}
\CommentTok{\#ggplot(cleaned\_coffee, aes(x=color)) + geom\_bar() + labs(title="Bar Chart of Color")}
\end{Highlighting}
\end{Shaded}

\begin{Shaded}
\begin{Highlighting}[]
\CommentTok{\#map(subset\_coffee, \textasciitilde{}sum(is.na(.)))}
\end{Highlighting}
\end{Shaded}

Our overarching question is: how does the aftertaste of coffee affect
satisfaction?

First, we plot a correlation matrix to see the relationship between each
pair of columns in the data.

\begin{Shaded}
\begin{Highlighting}[]
\CommentTok{\# Correlation Matrix }
\NormalTok{cleaned\_double\_coffee }\OtherTok{=}\NormalTok{ cleaned\_coffee[}\FunctionTok{c}\NormalTok{(}\StringTok{"total\_cup\_points"}\NormalTok{, }\StringTok{"aroma"}\NormalTok{, }\StringTok{"aftertaste"}\NormalTok{, }\StringTok{"acidity"}\NormalTok{, }\StringTok{"body"}\NormalTok{, }\StringTok{"balance"}\NormalTok{, }\StringTok{"sweetness"}\NormalTok{)]}
\FunctionTok{cor}\NormalTok{(cleaned\_double\_coffee)}
\end{Highlighting}
\end{Shaded}

\begin{verbatim}
##                  total_cup_points      aroma aftertaste    acidity       body
## total_cup_points        1.0000000 0.66931783  0.8136102 0.67994729 0.64906558
## aroma                   0.6693178 1.00000000  0.6594509 0.57202780 0.54542467
## aftertaste              0.8136102 0.65945092  1.0000000 0.67080456 0.67224309
## acidity                 0.6799473 0.57202780  0.6708046 1.00000000 0.60576938
## body                    0.6490656 0.54542467  0.6722431 0.60576938 1.00000000
## balance                 0.7642261 0.58432150  0.7560574 0.63141953 0.68504816
## sweetness               0.4522062 0.07435127  0.1353185 0.08080553 0.06776468
##                    balance  sweetness
## total_cup_points 0.7642261 0.45220618
## aroma            0.5843215 0.07435127
## aftertaste       0.7560574 0.13531850
## acidity          0.6314195 0.08080553
## body             0.6850482 0.06776468
## balance          1.0000000 0.11962640
## sweetness        0.1196264 1.00000000
\end{verbatim}

We see that most of the columns are somewhat correlated to the total cup
points, our outcome variable.

\begin{Shaded}
\begin{Highlighting}[]
\CommentTok{\#table(cleaned\_coffee$processing\_method)}

\CommentTok{\# Keeping only "Washed / Wet" processing method}
 \CommentTok{\#cleaned\_coffee = cleaned\_coffee[cleaned\_coffee$processing\_method=="Washed / Wet",]}


\NormalTok{cleaned\_coffee }\OtherTok{\textless{}{-}}\NormalTok{ cleaned\_coffee }\SpecialCharTok{\%\textgreater{}\%}
    \FunctionTok{mutate}\NormalTok{(}
    \AttributeTok{processing\_washed\_or\_wet =}  \FunctionTok{case\_when}\NormalTok{(}
\NormalTok{      processing\_method }\SpecialCharTok{==} \StringTok{"Washed / Wet"} \SpecialCharTok{\textasciitilde{}} \DecValTok{1}\NormalTok{,}
\NormalTok{      processing\_method }\SpecialCharTok{\%in\%} \FunctionTok{c}\NormalTok{(}\StringTok{"Natural / Dry"}\NormalTok{, }\StringTok{"Pulped natural / honey"}\NormalTok{, }\StringTok{"Semi{-}washed / Semi{-}pulped"}\NormalTok{, }\StringTok{"Other"}\NormalTok{) }\SpecialCharTok{\textasciitilde{}} \DecValTok{0}\NormalTok{,}
\NormalTok{    ))}

\CommentTok{\#head(cleaned\_coffee)}
\end{Highlighting}
\end{Shaded}

We plotted the relationships between pairs of variables, and found that
we don't need any drastic transformations.

\begin{Shaded}
\begin{Highlighting}[]
\FunctionTok{plot}\NormalTok{(cleaned\_coffee}\SpecialCharTok{$}\NormalTok{aftertaste, cleaned\_coffee}\SpecialCharTok{$}\NormalTok{total\_cup\_points)}
\end{Highlighting}
\end{Shaded}

\includegraphics{Lab2_Prelim_files/figure-latex/unnamed-chunk-12-1.pdf}

We are trying to investigate the effect of aftertaste on the total cup
points. We first build a linear model that has just the variable of
interest and the outcome variable.

\hypertarget{modeling}{%
\subsection{Modeling}\label{modeling}}

\begin{Shaded}
\begin{Highlighting}[]
\NormalTok{model\_1 }\OtherTok{=} \FunctionTok{lm}\NormalTok{(total\_cup\_points }\SpecialCharTok{\textasciitilde{}}\NormalTok{ aftertaste, }\AttributeTok{data=}\NormalTok{cleaned\_coffee)}
\CommentTok{\#summary(model\_1)}
\CommentTok{\# Coefficient interpretation: For a 1 unit increase in aftertaste, we expect total\_cup\_points to increase by about 6.47. We chose to examine the effect of aftertaste on our outcome variable because aftertaste has the highest correlation with total\_cup\_points.}
\end{Highlighting}
\end{Shaded}

\begin{Shaded}
\begin{Highlighting}[]
 \FunctionTok{stargazer}\NormalTok{(}
\NormalTok{   model\_1, }
   \AttributeTok{type =} \StringTok{\textquotesingle{}text\textquotesingle{}}\NormalTok{, }
   \AttributeTok{se =} \FunctionTok{list}\NormalTok{(}\FunctionTok{get\_robust\_se}\NormalTok{(model\_1))}
\NormalTok{   )}
\end{Highlighting}
\end{Shaded}

\begin{verbatim}
## 
## ===============================================
##                         Dependent variable:    
##                     ---------------------------
##                          total_cup_points      
## -----------------------------------------------
## aftertaste                   6.471***          
##                               (0.260)          
##                                                
## Constant                     34.268***         
##                               (1.952)          
##                                                
## -----------------------------------------------
## Observations                   1,044           
## R2                             0.662           
## Adjusted R2                    0.662           
## Residual Std. Error      1.533 (df = 1042)     
## F Statistic         2,040.489*** (df = 1; 1042)
## ===============================================
## Note:               *p<0.1; **p<0.05; ***p<0.01
\end{verbatim}

We see that aftertaste is significant in determining total cup points.

\begin{Shaded}
\begin{Highlighting}[]
\CommentTok{\#plot(model\_1)}
\end{Highlighting}
\end{Shaded}

Next, we build another model that includes other factors we suspect will
influence total cup points besides aftertaste. The covariates include
aroma, acidity, body, and balance.

\begin{Shaded}
\begin{Highlighting}[]
\CommentTok{\# Model 2: adding covariates}
\NormalTok{model\_2 }\OtherTok{=} \FunctionTok{lm}\NormalTok{(total\_cup\_points }\SpecialCharTok{\textasciitilde{}}\NormalTok{ aftertaste }\SpecialCharTok{+}\NormalTok{ aroma }\SpecialCharTok{+}\NormalTok{ acidity }\SpecialCharTok{+}\NormalTok{ body }\SpecialCharTok{+}\NormalTok{ balance, }\AttributeTok{data=}\NormalTok{cleaned\_coffee)}
\CommentTok{\#summary(model\_2)}
\end{Highlighting}
\end{Shaded}

\begin{Shaded}
\begin{Highlighting}[]
 \FunctionTok{stargazer}\NormalTok{(}
\NormalTok{   model\_2, }
   \AttributeTok{type =} \StringTok{\textquotesingle{}text\textquotesingle{}}\NormalTok{, }
   \AttributeTok{se =} \FunctionTok{list}\NormalTok{(}\FunctionTok{get\_robust\_se}\NormalTok{(model\_2))}
\NormalTok{   )}
\end{Highlighting}
\end{Shaded}

\begin{verbatim}
## 
## ===============================================
##                         Dependent variable:    
##                     ---------------------------
##                          total_cup_points      
## -----------------------------------------------
## aftertaste                   3.151***          
##                               (0.321)          
##                                                
## aroma                        1.390***          
##                               (0.248)          
##                                                
## acidity                      1.194***          
##                               (0.308)          
##                                                
## body                           0.321           
##                               (0.257)          
##                                                
## balance                      2.069***          
##                               (0.313)          
##                                                
## Constant                     21.358***         
##                               (1.850)          
##                                                
## -----------------------------------------------
## Observations                   1,044           
## R2                             0.744           
## Adjusted R2                    0.743           
## Residual Std. Error      1.337 (df = 1038)     
## F Statistic          603.216*** (df = 5; 1038) 
## ===============================================
## Note:               *p<0.1; **p<0.05; ***p<0.01
\end{verbatim}

In the results we see that the coefficient for aftertaste decreased,
which indicates that the covariates absorb some of the causal effect.
This is probably due to some causal effect between some of the
covariates and aftertaste.

\begin{Shaded}
\begin{Highlighting}[]
\CommentTok{\#plot(model\_2)}
\end{Highlighting}
\end{Shaded}

Lastly, we make a model with an interaction term that investigates
whether the processing method of the coffee beans affect aftertaste. By
looking at the coefficient, we should be able to tell if the aftertaste
is heterogenous whether coffee is processed wet or dry.

\begin{Shaded}
\begin{Highlighting}[]
\CommentTok{\# Model 3: include an interaction term}
\NormalTok{model\_3 }\OtherTok{=} \FunctionTok{lm}\NormalTok{(total\_cup\_points }\SpecialCharTok{\textasciitilde{}}\NormalTok{ aftertaste }\SpecialCharTok{+}\NormalTok{ aroma }\SpecialCharTok{+}\NormalTok{ acidity }\SpecialCharTok{+}\NormalTok{ body }\SpecialCharTok{+}\NormalTok{ balance }\SpecialCharTok{+}\NormalTok{ processing\_washed\_or\_wet }\SpecialCharTok{+}\NormalTok{ processing\_washed\_or\_wet }\SpecialCharTok{*}\NormalTok{ aftertaste , }\AttributeTok{data=}\NormalTok{cleaned\_coffee)}
\CommentTok{\#summary(model\_3)}

\CommentTok{\# The effect of aftertaste is consistent/homogenous regardless of using the processing method "Washed / Wet"  (there is no significant effect)}
\end{Highlighting}
\end{Shaded}

\begin{Shaded}
\begin{Highlighting}[]
 \FunctionTok{stargazer}\NormalTok{(}
\NormalTok{   model\_3, }
   \AttributeTok{type =} \StringTok{\textquotesingle{}text\textquotesingle{}}\NormalTok{, }
   \AttributeTok{se =} \FunctionTok{list}\NormalTok{(}\FunctionTok{get\_robust\_se}\NormalTok{(model\_3))}
\NormalTok{   )}
\end{Highlighting}
\end{Shaded}

\begin{verbatim}
## 
## ===============================================================
##                                         Dependent variable:    
##                                     ---------------------------
##                                          total_cup_points      
## ---------------------------------------------------------------
## aftertaste                                   3.285***          
##                                               (0.535)          
##                                                                
## aroma                                        1.369***          
##                                               (0.250)          
##                                                                
## acidity                                      1.143***          
##                                               (0.297)          
##                                                                
## body                                           0.367           
##                                               (0.261)          
##                                                                
## balance                                      2.080***          
##                                               (0.316)          
##                                                                
## processing_washed_or_wet                       1.129           
##                                               (4.660)          
##                                                                
## aftertaste:processing_washed_or_wet           -0.129           
##                                               (0.619)          
##                                                                
## Constant                                     20.351***         
##                                               (4.659)          
##                                                                
## ---------------------------------------------------------------
## Observations                                   1,044           
## R2                                             0.745           
## Adjusted R2                                    0.743           
## Residual Std. Error                      1.336 (df = 1036)     
## F Statistic                          432.031*** (df = 7; 1036) 
## ===============================================================
## Note:                               *p<0.1; **p<0.05; ***p<0.01
\end{verbatim}

We observed that the coefficient for the interaction term changed
slightly. This means that there is an effect (albeit small) of
processing method on the aftertaste. This intuitively makes sense, as
cleaning dirt or foreign objects from the coffee beans will improve
taste.

\begin{Shaded}
\begin{Highlighting}[]
\CommentTok{\#plot(model\_3)}
\end{Highlighting}
\end{Shaded}

\hypertarget{results}{%
\subsection{Results}\label{results}}

We found that aftertaste is statistically significant in determining the
the rating of a coffee. Thus it is important for farmers of coffee beans
to improve aftertaste if they want to receive higher scores, which will
command higher prices. Farmers can pay attention to the processing
method of coffee beans, which we found affects aftertaste.

It is important to recognize that aftertaste determines rating by a
relatively large margin, followed by balance of the coffee. The
appearence or ``body'' of the bean is the least important in determining
rating.

\hypertarget{limitations}{%
\subsection{Limitations}\label{limitations}}

(Statistical Limitation) One of our model assumptions is that our data
is IID. Since each rating for a coffee bean may come from a different
certification, the ratings come from different distributions. As the
chart below shows, Specialty Coffee Association has the highest number
of ratings given.

\begin{Shaded}
\begin{Highlighting}[]
\FunctionTok{ggplot}\NormalTok{(all\_ratings, }\FunctionTok{aes}\NormalTok{(}\AttributeTok{y=}\NormalTok{certification\_body)) }\SpecialCharTok{+} \FunctionTok{geom\_bar}\NormalTok{() }\SpecialCharTok{+} \FunctionTok{labs}\NormalTok{(}\AttributeTok{title=}\StringTok{"distribution of certification\_body"}\NormalTok{)}
\end{Highlighting}
\end{Shaded}

\includegraphics{Lab2_Prelim_files/figure-latex/unnamed-chunk-22-1.pdf}

\end{document}
